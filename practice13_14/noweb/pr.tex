\documentclass{article}\usepackage{noweb}\pagestyle{noweb}\noweboptions{}\begin{document}\nwfilename{pr.nw}\nwbegindocs{0}\section*{Быстрая сортировка}% ===> this file was generated automatically by noweave --- better not edit it

\footnote{Разработал Акимов В.Е.}

Рассмотрим реализацию одной из самых популярных сортировок на примере языка \texttt{Java}.

\nwenddocs{}\nwbegincode{1}\moddef{sort.java}\endmoddef\nwstartdeflinemarkup\nwenddeflinemarkup
public static void main(String[] args) \{
        \LA{}Исходный массив\RA{}
        \LA{}Вызов функции сортировки\RA{}
        \LA{}Вывод результата\RA{}
\}
\nwendcode{}\nwbegindocs{2}\nwdocspar



Исходный массив, который необходимо отсортировать.

\nwenddocs{}\nwbegincode{3}\moddef{Исходный массив}\endmoddef\nwstartdeflinemarkup\nwenddeflinemarkup
int[] testArray = \{32, -3, 99, 71, 1, -5, 0, 8, 44\};
\nwendcode{}\nwbegindocs{4}\nwdocspar



Вызов функции.

\nwenddocs{}\nwbegincode{5}\moddef{Вызов функции сортировки}\endmoddef\nwstartdeflinemarkup\nwenddeflinemarkup
quickSort(0, testArray.length - 1, testArray);
\nwendcode{}\nwbegindocs{6}\nwdocspar



Реализация функции сортировки.

\nwenddocs{}\nwbegincode{7}\moddef{sort.java}\plusendmoddef\nwstartdeflinemarkup\nwenddeflinemarkup
public static void quickSort(int begin, int end, int[] array) \{
        \LA{}Проверка\RA{}
        \LA{}Опорный элемент\RA{}
        \LA{}Создание переменных\RA{}
        \LA{}Цикл\RA{}
        
        \LA{}Сортировка слева\RA{}
        \LA{}Сортировка справа\RA{}
\}
\nwendcode{}\nwbegindocs{8}\nwdocspar



Проверка на длину исходного массива и на совпадение левой и правой его границы.

\nwenddocs{}\nwbegincode{9}\moddef{Проверка}\endmoddef\nwstartdeflinemarkup\nwenddeflinemarkup
if (array.length == 0 || end <= begin) \{
        return;
\}
\nwendcode{}\nwbegindocs{10}\nwdocspar



Опорный элемент для быстрой сортировки может быть любым. В данном примере он находится в середине массива для того, чтобы увеличить скорость сортировки.

\nwenddocs{}\nwbegincode{11}\moddef{Опорный элемент}\endmoddef\nwstartdeflinemarkup\nwenddeflinemarkup
int pivot = array[(begin + end) / 2];
\nwendcode{}\nwbegindocs{12}\nwdocspar



Создадим переменные для прохождения по элементам массива.

\nwenddocs{}\nwbegincode{13}\moddef{Создание переменных}\endmoddef\nwstartdeflinemarkup\nwenddeflinemarkup
int b = begin, e = end;
\nwendcode{}\nwbegindocs{14}\nwdocspar



Суть алгоритма быстрой сортировки заключается в разбиении массива на две части по опорному элементу и перемещении элементов, меньших опорному справой стороны на левую, а больших с левой на правую (при сортировки по возрастанию). Для этого необходим цикл, заданный ниже.

\nwenddocs{}\nwbegincode{15}\moddef{Цикл}\endmoddef\nwstartdeflinemarkup\nwenddeflinemarkup
while (b <= e) \{
        \LA{}Нахождение индекса большего элемента\RA{}
        \LA{}Нахождение индекса меньшего элемента\RA{}
        \LA{}Обмен\RA{}
\}
\nwendcode{}\nwbegindocs{16}\nwdocspar



Изначально необходимо найти элемент (его индекс), который больше опорного для переноса его с левой части на правую. 

\nwenddocs{}\nwbegincode{17}\moddef{Нахождение индекса большего элемента}\endmoddef\nwstartdeflinemarkup\nwenddeflinemarkup
while (array[b] < pivot) \{
        b++;
\}
\nwendcode{}\nwbegindocs{18}\nwdocspar



Аналогично для элемента, меньшего опорного.

\nwenddocs{}\nwbegincode{19}\moddef{Нахождение индекса меньшего элемента}\endmoddef\nwstartdeflinemarkup\nwenddeflinemarkup
while (array[e] > pivot) \{
        e--;
\}
\nwendcode{}\nwbegindocs{20}\nwdocspar



После нахождения этих индексов необходимо поменять местами эти элементы и изменить переменные для дальнейшего прохода по циклу.

\nwenddocs{}\nwbegincode{21}\moddef{Обмен}\endmoddef\nwstartdeflinemarkup\nwenddeflinemarkup
if (b <= e) \{
        int temp = array[b];
        array[b++] = array[e];
        array[e--] = temp;
\}
\nwendcode{}\nwbegindocs{22}\nwdocspar



Если переменная \texttt{e} будет больше, чем начальная позиция массива, что говорит о том, что с левой стороны относительно опорного элемента имеются элементы, то необходимо рекурсивно вызвать функцию, передав необходимые параметры.

\nwenddocs{}\nwbegincode{23}\moddef{Сортировка слева}\endmoddef\nwstartdeflinemarkup\nwenddeflinemarkup
if (begin < e) \{
        quickSort(begin, e, array);
\}
\nwendcode{}\nwbegindocs{24}\nwdocspar



Аналогичная проверка для переменной \texttt{b} относительно конечной позиции массива и вызов функции рекурсивно.

\nwenddocs{}\nwbegincode{25}\moddef{Сортировка справа}\endmoddef\nwstartdeflinemarkup\nwenddeflinemarkup
if (end > b) \{
        quickSort(b, end, array);
\}
\nwendcode{}\nwbegindocs{26}\nwdocspar



Для вывода отсортированного массива воспользуемся \texttt{Stream API}.

\nwenddocs{}\nwbegincode{27}\moddef{Вывод результата}\endmoddef\nwstartdeflinemarkup\nwenddeflinemarkup
Arrays.stream(testArray).forEach(el -> System.out.print(el + " "));
\nwendcode{}\nwbegindocs{28}\nwdocspar
\nwenddocs{}\end{document}

